\documentclass[11pt]{article}

\usepackage{times}
\usepackage{hyperref}
\usepackage{amsmath}
\usepackage{amssymb}
\usepackage{geometry}
\geometry{margin=1in}

\title{XEQUES: Proof-of-Command-Correctness for\\
Post-Quantum Autonomous Infrastructure}

\author{Prajwal Aher}

\date{}

\begin{document}
\maketitle

\begin{abstract}
We introduce XEQUES, a Layer-1 trust protocol based on
\emph{Proof-of-Command-Correctness (PoCC)}.
Unlike blockchains that finalize ordering, XEQUES finalizes
\emph{execution} by requiring autonomous devices to emit
cryptographically verifiable execution receipts.
The protocol is designed for post-quantum security,
offline verification, and long-lived autonomous systems.
\end{abstract}

\section{Introduction}
Modern autonomous infrastructure relies on cryptographic
authorization of commands but lacks cryptographic proof
that commands were actually executed.
XEQUES addresses this gap by making execution itself
a verifiable primitive.

\section{Threat Model}
We assume adversaries with full network visibility,
message replay capability, and access to quantum computation.
Adversaries may compromise authorities but cannot forge
device execution receipts.

\section{Cryptographic Foundations}
XEQUES relies on post-quantum digital signatures,
collision-resistant hashing, and monotonic sequencing.
Devices possess long-lived post-quantum identities.

\section{Protocol Flow}
Commands are authorized, executed, and finalized
only when a device emits a signed execution receipt.
No global consensus participation is required.

\section{Formal Invariants}
XEQUES enforces execution authenticity, non-repudiation,
immutability of receipts, and offline verifiability.

\section{Security Discussion}
We discuss replay resistance, authority compromise,
and long-term cryptographic robustness.

\section{Conclusion}
XEQUES defines a new trust primitive for autonomous systems:
cryptographically verifiable execution.
Future work includes formal proofs and deployment profiles.

\bibliographystyle{plain}
\end{document}
