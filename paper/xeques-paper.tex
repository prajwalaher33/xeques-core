\documentclass[11pt]{article}
\usepackage{amsmath}
\usepackage{amssymb}
\usepackage{hyperref}

\title{XEQUES: Proof-of-Command-Correctness}
\author{Prajwal Aher}
\date{\today}

\begin{document}
\maketitle

\begin{abstract}
XEQUES introduces Proof-of-Command-Correctness (PoCC), a Layer-1
trust primitive where autonomous devices emit cryptographically
verifiable execution receipts after executing authorized commands.
\end{abstract}

\section{Execution Semantics}
Devices are modeled as deterministic state machines.

\section{Receipt Definition}
Receipts bind command, state transition, and execution effects.

\section{Safety Lemma}
Receipt forgery implies compromise of the device private key.

\end{document}

\section{Formal Invariants}

\subsection{Invariant 1: Execution Authenticity}

\textbf{Implication:} A valid receipt cannot be produced without execution.

\subsection{Invariant 2: Non-Equivocation}
A device MUST NOT produce two distinct receipts for the same
command sequence number that imply different execution effects.

\textbf{Implication:} Conflicting histories are cryptographically detectable.

\subsection{Invariant 3: Causal Ordering}
Receipts encode a monotonic state transition.

\subsection{Invariant 4: Offline Verifiability}
Receipt verification requires only public data:
the command, the receipt, and the device public key.

\textbf{Implication:} Auditors can verify execution correctness
without device access or network connectivity.
